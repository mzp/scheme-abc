\documentclass{article}
\usepackage[dvipdfm]{graphicx, color}
\usepackage[utf8]{inputenc}
%\usepackage{fancysection}
\usepackage{fontspec}
\usepackage{xunicode}
\usepackage{xltxtra}



\XeTeXlinebreaklocale ``ja_JP''
\XeTeXlinebreakskip=0em plus 0.1em minus 0.01em
\XeTeXlinebreakpenalty=0
\XeTeX

\usepackage{hevea}

\begin{document}
\title{SchemeABC Reference 0.5.0}
\author{MIZUNO Hiroki}
\date{2009/05/27}
\maketitle
\tableofcontents

\section{Quick Start}
\subsection{動作環境}

ビルドには以下の環境が必要です。
\begin{itemize}
\item \footahref{http://caml.inria.fr/}{OCaml 3.11}以降
\item \footahref{http://www.camlcity.org/archive/programming/findlib.html}{Findlib}
\item \footahref{http://omake.metaprl.org/}{omake}
\item cpp
\item \footahref{http://hevea.inria.fr/}{hevea}
\end{itemize}

また以下のライブラリが必要になります。
\begin{itemize}
\item \footahref{http://code.google.com/p/ocaml-extlib/}{extlib}
\item \footahref{http://tech.motion-twin.com/xmllight.html}{xml-light}
\item \footahref{http://www.xs4all.nl/~mmzeeman/ocaml/ounit-doc/OUnit.html}{OUnit}
\end{itemize}

動作時には以下のソフトウェアが必要です。
\begin{itemize}
\item m4
\item \footahref{http://swfmill.org/}{swfmill}(svn trunk版)
\end{itemize}

開発は以下のプラットフォームで行なっています。Windowsでも動作することを目指していますが、現状ではいくつかの点で不安定です。
\begin{itemize}
\item MacOS X 10.5.7
\item Ubuntu 9.04
\end{itemize}


\subsection{ビルド方法}
一般的なシステムの場合、以下のコマンドでインストールが可能です。
\begin{verbatim}
$ omake config
$ omake all
$ sudo omake install
\end{verbatim}

インストール先のディレクトリを変更したい場合は、\verb!omake config!で指定します。

\begin{verbatim}
$ omake config PREFIX=/path/to
\end{verbatim}

\verb!PREFIX!以外にも表\ref{option}に示すオプションが利用可能です。

\begin{table}
\centering
\caption{利用可能なオプション}\label{option}
\begin{tabular}{|l|l|l|}
\hline
オプション名     & 説明 & デフォルト値 \\\hline
\verb!PREFIX!    & インストール先ディレクトリ & /usr/local \\
\verb!LIB_DIR!   & ライブラリのインストール先 & \verb!$PREFIX/lib/habc! \\
\verb!SHARE_DIR! & 共有ファイルのインストール先 & \verb!$PREFIX/share/habc! \\
\verb!BIN_DIR!   & コマンドのインストール先     & \verb!$PREFIX/bin! \\
\hline
\end{tabular}
\end{table}

\subsection{Hello,world}

``Hello,world''を表示するプログラムは以下のようになります。
\begin{verbatim}
(define-class Main (flash.display.Sprite) ())

(define-method init ([self Main])
  (let [(t (new flash.text.TextField))]
    (. t (appendText "Hello,world!!"))
    (. self (addChild t))))
\end{verbatim}

これをビルドします。
\begin{verbatim}
$ habc --height=100 --width=100 -o hello example/swf.scm
\end{verbatim}

生成された\verb!hello.swf!を適当なブラウザで開くと、図\ref{hello}が表示されます。

\begin{figure}
\centering
%BEGIN IMAGE
\includegraphics{hello.png}
%END IMAGE
%HEVEA \imageflush
\caption{Hello}\label{hello}
\end{figure}


\section{式}
\section{文}

\section{オブジェクトシステム}
CLOSライクです。

\section{標準ライブラリ}
まだありません。

\section{ライセンス}
\verb!pa_oo.ml! is written by Jacques Garrigue. and modified by OGASAWARA Satoshi., license is BSD.

\verb!pa_openin.ml! is written by Alain Frisch., licence is Public Domain.

Other codes is written by MIZUNO Hiroki., licence is MIT Licence.

\end{document}
