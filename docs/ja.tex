\documentclass{article}
\usepackage[utf8]{inputenc}

\usepackage{fancysection}

\begin{document}
\title{SchemeABC Reference 0.5.0}
\author{MIZUNO Hiroki}
\maketitle
\tableofcontents

\section{Quick Start}
\subsection{動作環境}

ビルドには以下の環境が必要です。
\begin{itemize}
\item OCaml 3.11以降
\item Findlib
\item omake
\item cpp
\item hevea
\end{itemize}

また以下のライブラリが必要になります。
\begin{itemize}
\item extlib
\item xml-light
\item OUnit
\end{itemize}

動作時には以下のソフトウェアが必要です。
\begin{itemize}
\item m4
\item swfmill(svn trunk版)
\end{itemize}

\subsection{ビルド方法}
一般的なシステムの場合、以下のコマンドでインストールが可能です。
\begin{verbatim}
$ omake config
$ omake all
$ sudo omake install
\end{verbatim}

インストール先のディレクトリを変更したい場合は、\verb!omake config!で指定します。

\begin{verbatim}
$ omake config PREFIX=/path/to
\end{verbatim}

\verb!PREFIX!以外にも表\ref{option}に示すオプションが利用可能です。

\begin{table}
\centering
\caption{利用可能なオプション}\label{option}
\begin{tabular}{|l|l|l|}
オプション名  & 説明 & デフォルト値 \\\hline
PREFIX        & インストール先ディレクトリ & /usr/local \\
LIB_DIR       & ライブラリのインストール先 & \verb!$PREFIX/lib/habc! \\
SHARE_DIR     & 共有ファイルのインストール先 & \verb!$PREFIX/share/habc! \\
BIN_DIR       & コマンドのインストール先     & \verb!$PREFIX/bin!
\end{tabular}
\end{table}


\subsection{Hello,world}

\section{式}

\section{オブジェクトシステム}
CLOSライクです。

\section{標準ライブラリ}
まだありません。
\section{ライセンス}

\end{document}
